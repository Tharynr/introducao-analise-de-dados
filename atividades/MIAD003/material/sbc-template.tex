\documentclass[12pt]{article}

\usepackage{sbc-template}

\usepackage{graphicx,url}

\usepackage[brazil]{babel}   
%\usepackage[latin1]{inputenc}  
\usepackage[utf8]{inputenc}  

\sloppy

\title{Introdução à analise de dados\\ Atividade três}

\author{Felipe Carvalho\inst{1}, Felipe Menino\inst{2}}

\address{FATEC "Jessen Vidal" 
    \email{\{felipe.carvalho69, felipe.carlos\}@fatec.sp.gov.br}
% \nextinstitute
%   Department of Computer Science -- University of Durham\\
%   Durham, U.K.
% \nextinstitute
%   Departamento de Sistemas e Computação\\
%   Universidade Regional de Blumenal (FURB) -- Blumenau, SC -- Brazil
%   \email{\{nedel,flavio\}@inf.ufrgs.br, R.Bordini@durham.ac.uk,
%   jomi@inf.furb.br}
}

\begin{document} 

\maketitle

% \begin{abstract}
%   This meta-paper describes the style to be used in articles and short papers
%   for SBC conferences. For papers in English, you should add just an abstract
%   while for the papers in Portuguese, we also ask for an abstract in
%   Portuguese (``resumo''). In both cases, abstracts should not have more than
%   10 lines and must be in the first page of the paper.
% \end{abstract}
     
\begin{resumo} 
    A aplicação dos conceitos vistos no mini-curso de introdução a análise de dados é extremamente importante para a fixação dos conteúdos. Esta é uma lista de atividades para auxiliar o aluno na fixação de todo o conteúdo.
\end{resumo}

\section{Contexto}

Um \textbf{web scrap} foi realizado com dez mil aplicativos da \textbf{Google Play Store}. Estes dados estão disponibilizados no \textbf{Kaggle}.

O interessante deste conjunto é tentar fazer verificações dos tipos mais populares de aplicativos, para quem sabe, ao começar um novo aplicativo, você já saiba uma tendência a seguir.

Os dados estão dentro do diretório \textbf{MIAD003}, porém encorajamos você a acessar o site do \textbf{Kaggle} e descobrir mais sobre ele e os conjuntos de dados presentes lá.

O link é este: https://www.kaggle.com/lava18/google-play-store-apps

\section{Atividade}

Como demonstrado ao longo das atividades do curso, à analise de dados é basicamente uma atividade utilizada para responder perguntas através de diferentes métodos de tratamento de manipulação dos dados.

Com este conjunto de dados, responda as seguintes questões:

\begin{itemize}
    \item Qual a categoria de aplicativo mais comum na \textbf{Google Play Store} ?
    \item Qual a categoria dos aplicativos com maior média de avaliações ?
    \item Nos aplicativos com avaliações maiores que 4, qual de \textbf{reviews} ? Padrões altos de avaliação, podem ser vinculados a quantidade de \textit{reviews} ?
    \item A maior quantidade de \textit{downloads} pertence a qual aplicativo ?
    \item Há mais aplicativos pagos ou grátis neste conjunto de dados ?
\end{itemize}

% \section{Referências}
% Bibliographic references must be unambiguous and uniform.  We recommend giving
% the author names references in brackets, e.g. \cite{knuth:84},
% \cite{boulic:91}, and \cite{smith:99}.

% The references must be listed using 12 point font size, with 6 points of space
% before each reference. The first line of each reference should not be
% indented, while the subsequent should be indented by 0.5 cm.

% \bibliographystyle{sbc}
% \bibliography{sbc-template}

\end{document}
